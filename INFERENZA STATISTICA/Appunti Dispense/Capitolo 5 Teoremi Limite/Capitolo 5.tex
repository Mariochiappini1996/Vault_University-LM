\documentclass[article,a4paper]{article}

% --- PACHETTI ESSENZIALI ---
\usepackage[utf8]{inputenc}
\usepackage[T1]{fontenc}
\usepackage[italian]{babel}
\usepackage[a4paper, margin=1in]{geometry} % Imposta i margini della pagina

% --- PACHETTI PER LA MATEMATICA ---
\usepackage{amsmath}
\usepackage{amssymb} % Per \mathbb
\usepackage{amsthm} % Per gli ambienti di proposizione e proof
\usepackage{mathrsfs} % Per \mathfrak

% --- DEFINIZIONE DEGLI AMBIENTI ---
\newtheoremstyle{miostile}
  {\topsep} % space before
  {\topsep} % space after
  {\itshape} % body font
  {} % indent
  {\bfseries} % head font
  {.} % punctuation after head
  {.5em} % space after head
  {} % head spec
\theoremstyle{miostile}

% Imposta la numerazione delle definizioni per iniziare da 58
% (Assumendo che questa sia la continuazione dei capitoli precedenti)
\newtheorem{theorem}{Theorem}
\newtheorem{remark}[theorem]{Remark}
\setcounter{theorem}{57}

% Rridefiniamo l'ambiente proof in italiano
\renewcommand{\proofname}{Proof}

% Comandi personalizzati
\newcommand{\R}{\mathbb{R}}
\newcommand{\N}{\mathbb{N}}
\newcommand{\E}{\mathbb{E}}


% --- INIZIO DEL DOCUMENTO ---
\begin{document}

\section*{5 Teoremi Limite}

\begin{theorem}[Legge dei grandi numeri per sequenze uniformemente integrabili]
Sia $\{X_{n}\}$ una sequenza di variabili aleatorie con valor medio nullo, indipendenti e uniformemente integrabili. Allora
\[
\overline{X}_{n}=n^{-1}\sum_{i=1}^{n}X_{i}\rightarrow_{1}0.
\]
\end{theorem}

\begin{proof}
Notiamo innanzitutto che non abbiamo ipotizzato che queste variabili
abbiano momento secondo finito e limitato, altrimenti il risultato seguirebbe
banalmente dalla legge dei grandi numeri di Chebyshev.
Le ipotesi del Lemma
sono verificate, ad esempio, se le variabili sono identicamente distribuite. Possiamo scrivere
\[
X_{n}=X_{n}I_{\{|X_{N}|\le M\}}+X_{n}I_{\{|X_{N}|>M\}}
\]
perché
\[
=X_{n}^{\prime}+X_{n}^{\prime\prime}=X_{n}^{\prime}-E[X_{n}^{\prime}]+X_{n}^{\prime\prime}-E[X_{n}^{\prime\prime}] ,
\]
$E[X_{n}^{\prime}]+E[X_{n}^{\prime\prime}]=0$. Abbiamo
\[
\frac{1}{n}\sum_{i=1}^{n}X_{i}=\frac{1}{n}\sum_{i=1}^{n}\{X_{i}^{\prime}-E[X_{i}^{\prime}]\}+\frac{1}{n}\sum_{i=1}^{n}\{X_{i}^{\prime\prime}-E[X_{i}^{\prime\prime}]\}
\]
La prima sequenza aleatoria è la media aritmetica di variabili aleatore con valor
medio nullo e varianza uniformemente limitata da $M^{2}$ quindi va a zero in media
quadratica per la legge dei grandi numeri di Chebyshev.
Per la seconda parte
possiamo scrivere
\[
E[|\frac{1}{n}\sum_{i=1}^{n}\{X_{i}^{\prime\prime}-E[X_{i}^{\prime\prime}]\}|]\le2~\sup_{i=1,2,3,...}E[|X_{n}I_{\{|X_{N}|>M\}}|]
\]
per l'uniforme integrabilità, scegliendo M in modo appropriato.
\end{proof}

\begin{theorem}[Legge forte dei grandi numeri - Borel]
Sia $\{X_{n}\}_{n\in\mathbb{N}}$ una successione di variabili aleatorie indipendenti ed identicamente distribuite con valor
medio $\mathbb{E}[X]=\mu$ momento quarto finito $\mathbb{E}[X^{4}]=\mu_{4}$ Allora
\[
\overline{X}_{n}:=\frac{1}{n}\sum_{i=1}^{n}X_{i}\rightarrow_{c.c.}\mu .
\]
\end{theorem}

\begin{proof}
Senza perdita di generalità consideriamo $\mu=0;$ infatti
\[
\overline{X}_n \rightarrow_{c.c.} 0, \quad \overline{X}_n := \frac{1}{n}\sum_{i=1}^n (X_i - \mu)
\]
Notiamo innanzitutto che $\mathbb{E}[\overline{X}_{n}^{4}]=O(\frac{1}{n^{2}})$;
infatti
\begin{align*}
\mathbb{E}[\overline{X}_{n}^{4}] &= \frac{1}{n^{4}}\sum_{i_{1},i_{2},i_{3},i_{4}=1}^{n}\mathbb{E}[X_{i_{1}}X_{i_{2}}X_{i_{3}}X_{i_{4}}] \\
&= \frac{1}{n^{4}}\sum_{i=1}^{n}\mathbb{E}[X_{i}^{4}]+\frac{3}{n^{4}}\sum_{i_{1} \ne i_{2}=1}^{n}\mathbb{E}[X_{i_{1}}^{2}]\mathbb{E}[X_{i_{2}}^{2}] \\
&= \frac{n}{n^{4}}\mathbb{E}[X_{1}^{4}]+\frac{3n(n-1)}{n^{4}}\mathbb{E}[X_{1}^{2}]\mathbb{E}[X_{2}^{2}] \\
&= O(n^{-2}) .
\end{align*}
(Nota: $\mathbb{E}[X_{i_1}^2]\mathbb{E}[X_{i_2}^2]$ è $\mathbb{E}[X_{1}^2]\mathbb{E}[X_{2}^2]$ per i.i.d.)

La prima uguaglianza è dovuta al fatto che $\mathbb{E}[X_{i_{1}}X_{i_{2}}X_{i_{3}}X_{i_{4}}]=0$ se c'è almeno
un indice diverso da tutti gli altri;
basta quindi focalizzarsi sul caso in cui gli
indici siano tutti uguali, oppure uguali a coppie (diverse tra loro).
Ci sono n
termini del primo tipo, $3n(n-1)$ del secondo.
A questo punto, utilizzando la
disuguaglianza di Markov, per ogni $\epsilon>0$ si ha
\[
\sum_{n=1}^{\infty}Pr\{|\overline{X}_{n}|>\epsilon\} \le \sum_{n=1}^{\infty} \frac{\mathbb{E}[\overline{X}_n^4]}{\epsilon^4} \le Const\times\sum_{n=1}^{\infty}\frac{1}{n^{2}\epsilon^{4}}<\infty ,
\]
e la disuguaglianza è dimostrata.
(Nota: la disuguaglianza di Markov è $\mathbb{P}(|\overline{X}_n| > \epsilon) = \mathbb{P}(\overline{X}_n^4 > \epsilon^4) \le \mathbb{E}[\overline{X}_n^4]/\epsilon^4$)
\end{proof}

\begin{remark}
$E^{\prime}$ interessante notare come questa legge forte dei grandi numeri
sia stata enunciata per la prima volta all'inizio del novecento da Emile Borel
all'interno di problemi di Teoria dei Numeri.
In particolare, Borel la utilizzo
per dimostrare che i numeri tra [0,1], espressi in forma binaria, hanno quasi
sempre una proporzione uguale di 0 e di 1. Questo esclude automaticamente tutti
i numeri con espansione finita o periodica, cioè i razionali;
ma in realtà dimostra
che hanno misura nulla anche numeri di classi più ampie.
Paradossalmente, per
pochissimi numeri conosciuti è stato effettivamente dimostrato che hanno questa
struttura "normale".
\end{remark}

\end{document}