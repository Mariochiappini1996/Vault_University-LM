\documentclass[article,a4paper]{article}

% --- PACHETTI ESSENZIALI ---
\usepackage[utf8]{inputenc}
\usepackage[T1]{fontenc}
\usepackage[italian]{babel}
\usepackage[a4paper, margin=1in]{geometry} % Imposta i margAini della pagina

% --- PACHETTI PER LA MATEMATICA ---
\usepackage{amsmath}
\usepackage{amssymb} % Per \mathbb
\usepackage{amsthm} % Per gli ambienti di proposizione e proof
\usepackage{mathrsfs} % Per \mathfrak
\usepackage{cases} % Per l'ambiente cases

% --- DEFINIZIONE DEGLI AMBIENTI ---
\newtheoremstyle{miostile}
  {\topsep} % space before
  {\topsep} % space after
  {\itshape} % body font
  {} % indent
  {\bfseries} % head font
  {.} % punctuation after head
  {.5em} % space after head
  {} % head spec
\theoremstyle{miostile}

% Imposta la numerazione delle definizioni per iniziare da 53
% (Assumendo che questa sia la continuazione dei capitoli precedenti)
\newtheorem{definition}{Definition}
\newtheorem{remark}[definition]{Remark}
\newtheorem{lemma}[definition]{Lemma}
\setcounter{definition}{52}

% Rridefiniamo l'ambiente proof in italiano
\renewcommand{\proofname}{Proof}

% Comandi personalizzati
\newcommand{\R}{\mathbb{R}}
\newcommand{\N}{\mathbb{N}}
\newcommand{\E}{\mathbb{E}}


% --- INIZIO DEL DOCUMENTO ---
\begin{document}

\section*{4 Uniforme Integrabilità}
Abbiamo visto in precedenza che la convergenza in probablità può implicare la
convergenza in media r-esima per sequenze di variabili aleatorie uniformemente
limitate, perché la limitatezza impedisce che con probabilità anche sempre più
piccola si assumano valori sempre più grandi.
E' naturale domandarsi se questa
richiesta di uniforme limitatezza non sia eccessiva;
dovrebbe essere possibile
"compensare" il comportamento sulle code in modo da far sì che la massa corrispondente decada comunque a zero.
Questa è proprio l'intuizione dietro alla
nozione di Uniforme Integrabilità.

\begin{definition}
La sequenza di variabili aleatorie $\{X_{n}\}$ è detta Uniformemente
Integrabile se
\[
\lim_{M\rightarrow\infty}\sup_{n}E[|X_{n}|I_{\{|X_{n}|>M\}}]=0 .
\]
\end{definition}

\begin{remark}
Notiamo innanzitutto che il fatto che i momenti siano uniformemente limitati non è sufficiente a garantire l'uniforme integrabilità: basta infatti
considerare l'esempio visto precedentemente, con:
\[
X_n = \begin{cases} 0 & \text{con probabilità } 1-\frac{1}{n^{2}} \\ n^2 & \text{con probabilità } \frac{1}{n^{2}} \end{cases}
\]
Qui $E[X_{n}]=1$ per ogni n, quindi il valor medio è unifomemente limitato, però
\[
\lim_{M\rightarrow\infty}\sup_{n}E[|X_{n}|I_{\{|X_{n}|>M\}}]=1 ,
\]
quindi la sequenza non è uniformemente integrabile. D'altra parte si vede subito
che la limitatezza dei momenti è una condizione necessaria.
\end{remark}

\begin{lemma}[Condizione necessaria per uniforme integrabilità]
Se $\{X_{n}\}$ è uniformemente integrabile, $E[X_{n}]$ è uniformemente limitato.
\end{lemma}

\begin{proof}
Si scelga $M=M_{\epsilon}$ tale per cui
\[
\sup_{n}E[|X_{n}|I_{\{|X_{n}|>M\}}]<\epsilon
\]
allora
\[
E[|X_{n}|]=E[|X_{n}|I_{\{|X_{n}|\le M_{\epsilon}\}}]+E[|X_{n}|I_{\{|X_{n}|>M_{\epsilon}\}}]\le M_{\epsilon}+\epsilon,
\]
da cui segue
che il valor medio è limitato uniformemente in n..
\end{proof}

E' interessante enunciare due condizioni sufficienti per l'Uniforme Integrabilità.

\begin{lemma}[Condizioni sufficienti per l'uniforme integrabilità]
\begin{itemize}
    \item[a)] Se la sequenza $\{X_{n}\}$ è costituita da variabili aleatorie identicamente distribuite con valor medio finito, essa è uniformememente integrabile
    \item[b)] Se esiste $\eta>0$ tale per cui $E[|X_{n}|^{1+\eta}]<M_{\eta}<\infty$ per ogni n, allora la
sequenza è uniformemente integrabile
\end{itemize}
\end{lemma}

\begin{proof}
Per a), è sufficiente notare che
\[
\lim_{M\rightarrow\infty}\sup_{n}E[|X_{n}|I_{\{|X_{n}|>M\}}]=\lim_{M\rightarrow\infty}E[|X_{1}|I_{\{|X_{1}|>M\}}]=0
\]
per l'ipotesi del momento primo finito. Per la b), abbiamo che
\begin{align*}
\lim_{M\rightarrow\infty}\sup_{n}E[|X_{n}|I_{\{|X_{n}|>M\}}] &\le \lim_{M\rightarrow\infty}\sup_{n}E[\frac{|X_{n}|^{\eta}}{M^{\eta}}|X_{n}|I_{\{|X_{n}|>M\}}] \\
&= \lim_{M\rightarrow\infty}\frac{1}{M^{\eta}}\sup_{n}E[|X_{n}|^{1+\eta}I_{\{|X_{n}|>M\}}] \\
&\le \lim_{M\rightarrow\infty}\frac{M_{\eta}}{M^{\eta}}=0 .
\end{align*}
\end{proof}

\begin{lemma}
Sia $\{X_{n}\}$ una sequenza uniformemente integrabile. Allora $X_{n}\rightarrow_{p}$
X implica $X_{n}\rightarrow_{1}X.$
\end{lemma}

\begin{proof}
Basta scrivere
\[
E[|X_n - X|] \le E[|X_n - X_n I_{\{|X_n| \le M\}}|] + E[|X_n I_{\{|X_n| \le M\}} - X I_{\{|X| \le M\}}|] + E[|X I_{\{|X| \le M\}} - X|].
\]
(Nota: La formula nel PDF è poco chiara, $E[XX] \le E X-XM|]+E[X-XM] + E[XAM- ]$. Quella sopra è un'interpretazione plausibile della scomposizione con una variabile troncata $X_M$).

Il primo elemento può essere reso arbitrariamente piccolo per l'uniforme integrabilità, il secondo perché X ha necessariamente valor medio finito (dimostrarlo),
il terzo perché abbiamo a che fare con variabili limitate che convergono in probabilità, quindi possiamos fruttare il risultato dimostrato precedentemente.
\end{proof}

\end{document}