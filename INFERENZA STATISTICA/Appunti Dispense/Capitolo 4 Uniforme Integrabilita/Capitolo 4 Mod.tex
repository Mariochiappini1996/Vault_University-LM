\documentclass[article,a4paper]{article}

% --- PACHETTI ESSENZIALI ---
\usepackage[utf8]{inputenc}
\usepackage[T1]{fontenc}
\usepackage[italian]{babel}
\usepackage[a4paper, margin=1in]{geometry} % Imposta i margini della pagina

% --- PACHETTI PER LA MATEMATICA ---
\usepackage{amsmath}
\usepackage{amssymb} % Per \mathbb
\usepackage{amsthm} % Per gli ambienti di proposizione e proof
\usepackage{mathrsfs} % Per \mathfrak
\usepackage{cases} % Per l'ambiente cases

% --- PACCHETTO PER LE SPIEGAZIONI (TCOLORBOX) ---
\usepackage[skins,breakable]{tcolorbox}

% --- DEFINIZIONE DEGLI AMBIENTI ---
\newtheoremstyle{miostile}
  {\topsep} % space before
  {\topsep} % space after
  {\itshape} % body font
  {} % indent
  {\bfseries} % head font
  {.} % punctuation after head
  {.5em} % space after head
  {} % head spec
\theoremstyle{miostile}

% Imposta la numerazione delle definizioni per iniziare da 53
% (Assumendo che questa sia la continuazione dei capitoli precedenti)
\newtheorem{definition}{Definition}
\newtheorem{remark}[definition]{Remark}
\newtheorem{lemma}[definition]{Lemma}
\setcounter{definition}{52}

% Rridefiniamo l'ambiente proof in italiano
\renewcommand{\proofname}{Proof}

% Comandi personalizzati
\newcommand{\R}{\mathbb{R}}
\newcommand{\N}{\mathbb{N}}
\newcommand{\E}{\mathbb{E}}


% --- STILE PER LE SCATOLE DI SPIEGAZIONE ---
% Definiamo un nuovo ambiente tcolorbox chiamato 'spiegazione'
\tcbset{
    spiegazionestyle/.style={
        colback=gray!10, % Sfondo grigio chiaro
        colframe=gray!60, % Bordo grigio scuro
        fonttitle=\bfseries,
        title=Spiegazione Semplice,
        sharp corners,
        boxsep=5pt,
        left=5pt,
        right=5pt,
        top=5pt,
        bottom=5pt,
        breakable, % Permette al box di spezzarsi tra le pagine
    }
}
% Creiamo il comando \begin{spiegazione} ... \end{spiegazione}
\newtcolorbox{spiegazione}{spiegazionestyle}


% --- INIZIO DEL DOCUMENTO ---
\begin{document}

\section*{4 Uniforme Integrabilità}
Abbiamo visto in precedenza che la convergenza in probablità può implicare la
convergenza in media r-esima per sequenze di variabili aleatorie uniformemente
limitate, perché la limitatezza impedisce che con probabilità anche sempre più
piccola si assumano valori sempre più grandi.
E' naturale domandarsi se questa
richiesta di uniforme limitatezza non sia eccessiva;
dovrebbe essere possibile
"compensare" il comportamento sulle code in modo da far sì che la massa corrispondente decada comunque a zero.
Questa è proprio l'intuizione dietro alla
nozione di Uniforme Integrabilità.

\begin{spiegazione}
    \textbf{Qual è il problema che stiamo risolvendo?}

    Nei capitoli precedenti, abbiamo visto che la "Convergenza in Probabilità" (tipo 2) non garantisce la "Convergenza in Media" (tipo 3).
    
    Il problema è il contro-esempio: $X_n = n$ con probabilità $1/n$, e $0$ altrimenti.
    \begin{itemize}
        \item \textbf{Converge in probabilità a 0?} Sì. L'evento "sfortunato" ($X_n=n$) diventa sempre più raro (prob $1/n \to 0$).
        \item \textbf{Converge in media a 0?} No. La media $\mathbb{E}[|X_n|] = (n \times 1/n) + (0 \times \dots) = 1$. La media resta 1 per sempre.
    \end{itemize}
    
    Abbiamo visto che se $X_n$ è \textit{limitata} (es. non può mai superare 1000), questo problema sparisce. Ma "limitata" è un requisito troppo forte.
    
    L'\textbf{Uniforme Integrabilità (UI)} è un requisito più debole e più intelligente: non chiede che i valori "estremi" siano impossibili, ma solo che, in media, la loro influenza diventi trascurabile all'aumentare di $n$. È la condizione "giusta" per far funzionare le cose.
\end{spiegazione}

\begin{definition}
La sequenza di variabili aleatorie $\{X_{n}\}$ è detta Uniformemente
Integrabile se
\[
\lim_{M\rightarrow\infty}\sup_{n}E[|X_{n}|I_{\{|X_{n}|>M\}}]=0 .
\]
\end{definition}

\begin{spiegazione}
    \textbf{Decodifichiamo la formula}

    Analizziamo la formula pezzo per pezzo:
    \begin{itemize}
        \item \textbf{$I_{\{|X_{n}|>M\}}$}: È la "funzione indicatrice". Vale 1 se $X_n$ è "estremo" (cioè $|X_n| > M$), e 0 altrimenti.
        \item \textbf{$E[|X_{n}|I_{\{|X_{n}|>M\}}]$}: È il "contributo medio delle code". Calcoliamo la media di $X_n$, ma consideriamo \textit{solo} i valori estremi, più grandi di $M$. È una misura di "quanta massa" (quanta probabilità moltiplicata per il valore) sta nelle code della distribuzione.
        \item \textbf{$\sup_{n} \dots$}: Prendiamo il valore "peggiore" (il massimo, "supremum") di questo contributo medio su \textit{tutta} la successione $n=1, 2, \dots$.
        \item \textbf{$\lim_{M\rightarrow\infty} \dots = 0$}: Diciamo che la successione è UI se, scegliendo una soglia $M$ abbastanza grande (es. $M=1.000.000$), il contributo medio delle code (anche nel caso peggiore $n$) diventa praticamente zero.
    \end{itemize}
    In breve: \textbf{le code (i valori estremi) di tutte le $X_n$ contemporaneamente diventano trascurabili in media, se guardiamo abbastanza lontano (M grande).}
\end{spiegazione}

\begin{remark}
Notiamo innanzitutto che il fatto che i momenti siano uniformemente limitati non è sufficiente a garantire l'uniforme integrabilità: basta infatti
considerare l'esempio visto precedentemente, con:
\[
X_n = \begin{cases} 0 & \text{con probabilità } 1-\frac{1}{n^{2}} \\ n^2 & \text{con probabilità } \frac{1}{n^{2}} \end{cases}
\]
Qui $E[X_{n}]=1$ per ogni n, quindi il valor medio è unifomemente limitato, però
\[
\lim_{M\rightarrow\infty}\sup_{n}E[|X_{n}|I_{\{|X_{n}|>M\}}]=1 ,
\]
quindi la sequenza non è uniformemente integrabile. D'altra parte si vede subito
che la limitatezza dei momenti è una condizione necessaria.
\end{remark}

\begin{spiegazione}
    \textbf{Momenti limitati $\neq$ Uniforme Integrabilità}

    Questo contro-esempio è importante.
    \begin{itemize}
        \item \textbf{Media (momento primo) limitata?} Sì. La media è $\mathbb{E}[X_n] = (n^2 \times 1/n^2) + (0 \times \dots) = 1$. La media è sempre 1, quindi è "uniformemente limitata".
        \item \textbf{È Uniformemente Integrabile?} No. Se calcoliamo il contributo medio delle code (oltre una soglia $M$), per $n$ molto grande (tale che $n^2 > M$), il valore "estremo" $n^2$ viene sempre "catturato". Il contributo medio delle code sarà sempre 1 (per $n$ abbastanza grande). Il limite per $M \to \infty$ non va a 0, ma resta 1.
    \end{itemize}
    Questo dimostra che avere una media costantemente pari a 1 non basta per "controllare le code" nel modo specifico richiesto dall'UI.
\end{spiegazione}

\begin{lemma}[Condizione necessaria per uniforme integrabilità]
Se $\{X_{n}\}$ è uniformemente integrabile, $E[X_{n}]$ è uniformemente limitato.
\end{lemma}

\begin{proof}
Si scelga $M=M_{\epsilon}$ tale per cui
\[
\sup_{n}E[|X_{n}|I_{\{|X_{n}|>M\}}]<\epsilon
\]
allora
\[
E[|X_{n}|]=E[|X_{n}|I_{\{|X_{n}|\le M_{\epsilon}\}}]+E[|X_{n}|I_{\{|X_{n}|>M_{\epsilon}\}}]\le M_{\epsilon}+\epsilon,
\]
da cui segue
che il valor medio è limitato uniformemente in n..
\end{proof}

\begin{spiegazione}
    \textbf{UI $\implies$ Momenti limitati}

    Questo lemma dice il contrario del remark precedente. Se una successione è UI (cioè le sue code sono "controllate"), allora \textit{necessariamente} anche la sua media (momento primo) deve essere "controllata" (uniformemente limitata).
    
    La dimostrazione è intuitiva: la media totale $\mathbb{E}[|X_n|]$ viene divisa in due parti:
    \begin{enumerate}
        \item \textbf{La "pancia" ($|X_n| \le M$):} Questa parte è al massimo $M$ (un numero finito).
        \item \textbf{Le "code" ($|X_n| > M$):} Per l'ipotesi di UI, questa parte è piccolissima (minore di $\epsilon$).
    \end{enumerate}
    La media totale è al massimo $M + \epsilon$, un numero finito. Quindi è limitata.
\end{spiegazione}

\begin{lemma}[Condizioni sufficienti per l'uniforme integrabilità]
\begin{itemize}
    \item[a)] Se la sequenza $\{X_{n}\}$ è costituita da variabili aleatorie identicamente distribuite con valor medio finito, essa è uniformememente integrabile
    \item[b)] Se esiste $\eta>0$ tale per cui $E[|X_{n}|^{1+\eta}]<M_{\eta}<\infty$ per ogni n, allora la
sequenza è uniformemente integrabile
\end{itemize}
\end{lemma}

\begin{spiegazione}
    \textbf{Come faccio a sapere se una successione è UI?}

    Controllare la definizione è difficile. Questo lemma ci dà due scorciatoie (condizioni sufficienti):
    
    \begin{itemize}
        \item \textbf{a) Sono Identicamente Distribuite?} Se tutte le $X_n$ seguono la stessa identica distribuzione (es. sono tutti lanci dello stesso dado) e la loro media è finita (es. la media di un dado è 3.5), allora la successione è UI. Facile.
        
        \item \textbf{b) Hanno un "momento" più alto controllato?} Se le $X_n$ non sono identiche, ma puoi dimostrare che un "momento" (una media) \textit{leggermente superiore} al primo (es. $\mathbb{E}[|X_n|^{1.01}]$) è uniformemente limitato (cioè sta sotto un numero $M_\eta$ per ogni $n$), allora la successione è UI.
    \end{itemize}
    La condizione (b) è molto potente: "controllare" un momento superiore (come 1.5, o 2) è sufficiente per "schiacciare" le code e garantire l'UI.
\end{spiegazione}

\begin{proof}
Per a), è sufficiente notare che
\[
\lim_{M\rightarrow\infty}\sup_{n}E[|X_{n}|I_{\{|X_{n}|>M\}}]=\lim_{M\rightarrow\infty}E[|X_{1}|I_{\{|X_{1}|>M\}}]=0
\]
per l'ipotesi del momento primo finito. Per la b), abbiamo che
\begin{align*}
\lim_{M\rightarrow\infty}\sup_{n}E[|X_{n}|I_{\{|X_{n}|>M\}}] &\le \lim_{M\rightarrow\infty}\sup_{n}E[\frac{|X_{n}|^{\eta}}{M^{\eta}}|X_{n}|I_{\{|X_{n}|>M\}}] \\
&= \lim_{M\rightarrow\infty}\frac{1}{M^{\eta}}\sup_{n}E[|X_{n}|^{1+\eta}I_{\{|X_{n}|>M\}}] \\
&\le \lim_{M\rightarrow\infty}\frac{M_{\eta}}{M^{\eta}}=0 .
\end{align*}
\end{proof}

\begin{lemma}
Sia $\{X_{n}\}$ una sequenza uniformemente integrabile. Allora $X_{n}\rightarrow_{p}$
X implica $X_{n}\rightarrow_{1}X.$
\end{lemma}

\begin{spiegazione}
    \textbf{Il Teorema Fondamentale dell'UI}

    Questo è il risultato principale del capitolo. Ci dice che l'Uniforme Integrabilità (UI) è esattamente l'ingrediente che mancava.
    
    \textbf{Convergenza in Probabilità (2) + Uniforme Integrabilità (UI) $\implies$ Convergenza in Media r=1 (3)}
    
    Se sappiamo che $X_n$ converge a $X$ in probabilità (quindi i casi "sfortunati" diventano rari) E sappiamo anche che $X_n$ è UI (quindi i casi "sfortunati" non sono mai *troppo* disastrosi in media), allora possiamo concludere che $X_n$ converge a $X$ anche in media.
    
    L'UI "tappa il buco" che permetteva al contro-esempio ($X_n=n$ con prob $1/n$) di funzionare.
\end{spiegazione}

\begin{proof}
Basta scrivere
\[
E[|X_n - X|] \le E[|X_n - X_n I_{\{|X_n| \le M\}}|] + E[|X_n I_{\{|X_n| \le M\}} - X I_{\{|X| \le M\}}|] + E[|X I_{\{|X| \le M\}} - X|].
\]
Il primo elemento può essere reso arbitrariamente piccolo per l'uniforme integrabilità, il secondo perché X ha necessariamente valor medio finito (dimostrarlo),
il terzo perché abbiamo a che fare con variabili limitate che convergono in probabilità, quindi possiamos fruttare il risultato dimostrato precedentemente.
\end{proof}

\end{document}