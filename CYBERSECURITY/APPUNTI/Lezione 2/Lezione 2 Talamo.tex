% File: Lezione_Cybersecurity_Appunti.tex
% Formattato per Overleaf
\documentclass[11pt,a4paper]{article}
\usepackage[utf8]{inputenc}
\usepackage[T1]{fontenc}
\usepackage[italian]{babel}
\usepackage{geometry}
\usepackage{hyperref}
\usepackage{enumitem}
\usepackage{amsmath,amssymb}
\usepackage{tikz}
\geometry{margin=2.5cm}

\title{Appunti della lezione di Cybersecurity}
\author{Corso universitario — Lezione 3 Prof. Talamo}
\date{}

\begin{document}

\maketitle
\tableofcontents
\vspace{1cm}

\section{Registry e mappatura su DNS}
\begin{itemize}
  \item I registry virtualizzano risorse e si occupano della distribuzione degli stessi.
  \item Se un attaccante compromette un registry, può alterare l'associazione tra un'identità logica e il suo indirizzo fisico (es. record DNS) e quindi impadronirsi dell'identità di un soggetto agli occhi degli altri sistemi.
  \item Un attacco al registry si propaga su tutti gli oggetti che dipendono da quella mappatura.
  \item Per accedere al DNS è necessario un protocollo specifico: la sua implementazione e gestione diventano quindi superfici d'attacco critiche.
\end{itemize}

\section{Esempio: attacco a infrastruttura cloud (caso pratico)}
\begin{itemize}
  \item Caso esemplificativo: attacco che ha interessato un provider (es. scenario tipo Amazon).
  \item L'attaccante ha preso il controllo del registry virtuale: la risposta dell'operatore è stata chiudere il registry compromesso e aprirne uno nuovo.
  \item L'avversario si era invece posizionato sulle macchine virtuali, rendendo l'impatto più esteso e meno evidente; parte dell'utenza resta inconsapevole del danno subito.
  \item Domande critiche: perch\'e l'infrastruttura presenta questa architettura? Quale componente \'e stato effettivamente attaccato? (registry, hypervisor, VM, rete, ecc.)
\end{itemize}

\section{Architettura vs infrastruttura di sicurezza}
\begin{description}[leftmargin=!,labelwidth=3cm]
  \item[Architettura sistemistica] l'insieme dei componenti hardware e software e la loro organizzazione (es. macchine, hypervisor, rete, DB).
  \item[Infrastruttura della sicurezza] le soluzioni, le politiche e i processi che si adottano per proteggere i contesti rilevanti all'interno dell'architettura.
\end{description}

\section{Standard e protezione delle vulnerabilit\`a}
\begin{itemize}
  \item Negli anni 2000 si svilupparono iniziative in Europa e negli USA per progettare infrastrutture digitali sicure: l'obiettivo era progettare sistemi orientati alla protezione delle identit\`a e dei contesti.
  \item Una \emph{potenziale vulnerabilit\`a} viene qui definita come un "contesto" che contiene informazioni e che decidiamo di proteggere perch\'e distinguibile e identificabile.
  \item L'elemento comune e\` l'esistenza di un'identit\`a associata ad un contesto informativo; intorno a questo concetto nascono protocolli e linee guida (es. best practices, norme per il trattamento dati).
\end{itemize}

\section{Filosofia dei contesti e livello di astrazione}
\begin{itemize}
  \item L'approccio basato sui contesti introduce un livello di astrazione: invece di difendere singoli componenti isolati, si individuano contesti identificati (es. servizi, archivi, funzioni) e si progetta la difesa attorno a questi.
  \item Questo approccio \`e empirico e spesso non visibile nella progettazione iniziale dell'architettura.
\end{itemize}

\section{Esempio reale: gestione di un "Comune"}
Descriviamo un esempio pratico per capire la logica dei contesti.
\begin{itemize}
  \item \textbf{Contenitori/servizi presenti:}
  \begin{itemize}
    \item Comando vigile (archivio multe, dati anagrafici, archivio personale, integrazione videosorveglianza)
    \item Museo (archivi propri, biglietteria con carte di credito)
    \item Portale cittadino con accesso via login collegato a un database centrale
  \end{itemize}
  \item \textbf{Analisi del rischio e vettori d'attacco:}
  \begin{itemize}
    \item La sezione \emph{museo} pu\`o essere particolarmente critica per la presenza di dati di pagamento (biglietteria online): un attacco qui potrebbe portare a frodi finanziarie.
    \item Un attaccante che compromette il portale di accesso pu\`o potenzialmente raggiungere l'intero sistema se il portale funge da punto di ingresso comune al database centrale.
    \item La scelta di difendere solo una parte dell'infrastruttura (es. il "comando vigile") rischia di lasciare scoperti altri contesti pi\`u sensibili.
  \end{itemize}
  \item \textbf{Identit\`a del contesto:} l'identit\`a da proteggere non \`e solo username/password, ma l'identit\`a del contesto (es. il ticket + carta di credito nella biglietteria).
\end{itemize}

\section{Perimetro e superficie d'attacco}
\begin{itemize}
  \item Il perimetro \`e il confine che racchiude tutti i contesti appartenenti a una struttura (es. l'insieme dei servizi di un Comune).
  \item La superficie d'attacco \`e l'insieme dei punti di vulnerabilit\`a compresi all'interno di quel perimetro: punti di accesso, API, portali, database, servizi esterni collegati, ecc.
  \item Definire il perimetro aiuta a raggruppare contesti e a calcolare la superficie d'attacco in modo strutturato.
\end{itemize}

\section{Esercizio proposto}
\begin{enumerate}
  \item Scegliere un'architettura di riferimento (es. architettura database, architettura di servizio su cloud).
  \item Disegnare il perimetro: elencare i contesti che appartengono alla struttura.
  \item Per ogni contesto decidere quale sia la superficie d'attacco (vettori principali, dipendenze esterne, punti di ingresso).
  \item Applicare l'analisi a una piattaforma cloud moderna (es. Microsoft Azure): mappare i servizi Azure usati (VM, App Service, Azure AD, Database, Storage, Gateway) e valutare perimetro e superficie d'attacco.
\end{enumerate}

\vfill
\noindent---
\newline
% Fine del documento
\end{document}
